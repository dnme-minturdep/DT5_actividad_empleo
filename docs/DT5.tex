% Options for packages loaded elsewhere
\PassOptionsToPackage{unicode}{hyperref}
\PassOptionsToPackage{hyphens}{url}
%
\documentclass[
  openany]{book}
\usepackage{lmodern}
\usepackage{amsmath}
\usepackage{ifxetex,ifluatex}
\ifnum 0\ifxetex 1\fi\ifluatex 1\fi=0 % if pdftex
  \usepackage[T1]{fontenc}
  \usepackage[utf8]{inputenc}
  \usepackage{textcomp} % provide euro and other symbols
  \usepackage{amssymb}
\else % if luatex or xetex
  \usepackage{unicode-math}
  \defaultfontfeatures{Scale=MatchLowercase}
  \defaultfontfeatures[\rmfamily]{Ligatures=TeX,Scale=1}
\fi
% Use upquote if available, for straight quotes in verbatim environments
\IfFileExists{upquote.sty}{\usepackage{upquote}}{}
\IfFileExists{microtype.sty}{% use microtype if available
  \usepackage[]{microtype}
  \UseMicrotypeSet[protrusion]{basicmath} % disable protrusion for tt fonts
}{}
\makeatletter
\@ifundefined{KOMAClassName}{% if non-KOMA class
  \IfFileExists{parskip.sty}{%
    \usepackage{parskip}
  }{% else
    \setlength{\parindent}{0pt}
    \setlength{\parskip}{6pt plus 2pt minus 1pt}}
}{% if KOMA class
  \KOMAoptions{parskip=half}}
\makeatother
\usepackage{xcolor}
\IfFileExists{xurl.sty}{\usepackage{xurl}}{} % add URL line breaks if available
\IfFileExists{bookmark.sty}{\usepackage{bookmark}}{\usepackage{hyperref}}
\hypersetup{
  pdftitle={Documento Técnico N°5: Medición de la contribución económica del turismo: actividad y empleo},
  pdfauthor={Dirección Nacional de Mercados y Estadísticas - Subsecretaría de Desarrollo Estratégico},
  hidelinks,
  pdfcreator={LaTeX via pandoc}}
\urlstyle{same} % disable monospaced font for URLs
\usepackage{longtable,booktabs}
\usepackage{calc} % for calculating minipage widths
% Correct order of tables after \paragraph or \subparagraph
\usepackage{etoolbox}
\makeatletter
\patchcmd\longtable{\par}{\if@noskipsec\mbox{}\fi\par}{}{}
\makeatother
% Allow footnotes in longtable head/foot
\IfFileExists{footnotehyper.sty}{\usepackage{footnotehyper}}{\usepackage{footnote}}
\makesavenoteenv{longtable}
\usepackage{graphicx}
\makeatletter
\def\maxwidth{\ifdim\Gin@nat@width>\linewidth\linewidth\else\Gin@nat@width\fi}
\def\maxheight{\ifdim\Gin@nat@height>\textheight\textheight\else\Gin@nat@height\fi}
\makeatother
% Scale images if necessary, so that they will not overflow the page
% margins by default, and it is still possible to overwrite the defaults
% using explicit options in \includegraphics[width, height, ...]{}
\setkeys{Gin}{width=\maxwidth,height=\maxheight,keepaspectratio}
% Set default figure placement to htbp
\makeatletter
\def\fps@figure{htbp}
\makeatother
\setlength{\emergencystretch}{3em} % prevent overfull lines
\providecommand{\tightlist}{%
  \setlength{\itemsep}{0pt}\setlength{\parskip}{0pt}}
\setcounter{secnumdepth}{5}
  %%% REFERENCIAS
        \usepackage{hyperref}
        % links del indice en negro; citas y URL en azul
        \hypersetup{colorlinks = true, urlcolor={blue}, 
        citecolor={blue}, linkcolor ={blue}}
\usepackage[spanish]{babel} % Idiomas en los que se escribe el documento. 
\usepackage{booktabs}
\usepackage{amsthm}

\usepackage[final]{pdfpages}

\makeatletter
\def\thm@space@setup{%
  \thm@preskip=8pt plus 2pt minus 4pt
  \thm@postskip=\thm@preskip
}
\makeatother
\let\oldmaketitle\maketitle
\AtBeginDocument{\let\maketitle\relax}
\ifluatex
  \usepackage{selnolig}  % disable illegal ligatures
\fi
\usepackage[]{natbib}
\bibliographystyle{apalike}

\title{Documento Técnico N°5: Medición de la contribución económica del turismo: actividad y empleo}
\usepackage{etoolbox}
\makeatletter
\providecommand{\subtitle}[1]{% add subtitle to \maketitle
  \apptocmd{\@title}{\par {\large #1 \par}}{}{}
}
\makeatother
\subtitle{El turismo desde la perspectiva de la oferta}
\author{Dirección Nacional de Mercados y Estadísticas - Subsecretaría de Desarrollo Estratégico}
\date{21 de julio de 2021}

\begin{document}
\maketitle

\includepdf[pages={1}, scale=1]{DT5Portada.pdf}
\newpage

\let\maketitle\oldmaketitle
\maketitle

{
\setcounter{tocdepth}{1}
\tableofcontents
}
\hypertarget{presentaciuxf3n}{%
\chapter*{Presentación}\label{presentaciuxf3n}}
\addcontentsline{toc}{chapter}{Presentación}

El presente documento, \textbf{Medición de la contribución económica del turismo: actividad y empleo}, se enmarca en el proyecto de Armonización de las Estadísticas de Turismo en las Provincias de la \href{http://datos.yvera.gob.ar/}{Dirección Nacional de Mercados y Estadística de la Subsecretaría de Desarrollo Estratégico del Ministerio de Turismo y Deportes}. El objetivo general de este proyecto es contribuir con propuestas metodológicas para los sistemas de estadísticas de turismo provinciales que orienten a producir indicadores provinciales básicos y comparables.

Además de este, se encuentra disponible una serie de documentos técnicos que abordan otras problemáticas vinculadas a la producción de estadística de turismo:

\begin{itemize}
\item
  \href{https://dnme-minturdep.github.io/DT1_medicion_turismo/}{Documento Técnico \#1}: Conceptos y elementos básicos para la medición provincial de los turistas
\item
  \href{https://dnme-minturdep.github.io/DT2_encuestas/}{Documento Técnico \#2}: Propuestas metodológicas para las encuestas de ocupación en alojamientos turísticos
\item
  \href{https://dnme-minturdep.github.io/DT3_registros_adminsitrativos/}{Documento Técnico \#3}: Descripción, análisis y utilización de los Registros Administrativos para la medición del Turismo
\item
  \href{https://dnme-minturdep.github.io/DT4_perfiles/}{Documento Técnico \#4}: Propuestas Metodológicas para las Encuestas de Perfil del Visitante
\end{itemize}

\hypertarget{documento-tuxe9cnico-nuxba5---resumen}{%
\subsection*{Documento Técnico Nº5 - Resumen}\label{documento-tuxe9cnico-nuxba5---resumen}}
\addcontentsline{toc}{subsection}{Documento Técnico Nº5 - Resumen}

El objetivo general es describir la metodología utilizada para elaborar indicadores de la actividad del turismo desde la perspectiva de oferta. En particular se analiza los desafíos de la medición de la contribución económica del turismo a la actividad y el empleo desde las distintas ópticas territoriales. En virtud de estos objetivos, este documento se estructura en tres capítulos.

En el \textbf{Capítulo} \ref{medicion-actividad} se desarrolla una metodología, según las recomendaciones internacionales, para la medición de la contribución económica del turismo en un territorio. En sus diferentes secciones, presenta la definición y el alcance de las actividades económicas involucradas en el turismo a partir del enfoque de las Ramas Características, recomendaciones generales para la medición, indicadores macroeconómicos básicos que darán cuenta de la contribución desde la perspectiva de la oferta y antecedentes nacional e internacional del uso de esta perspectiva para la medición de la contribución económica del turismo en sus respectivos territorios de interés.

En el \textbf{Capítulo} \ref{medicion-empleo} se desarrollan las problemáticas básicas al momento de medir el empleo en el sector turístico como parte del sistema económico. En sus diferentes secciones, presenta una síntesis sobre las definiciones básicas y las recomendaciones internacionales para la medición del empleo y los antecedentes internacionales y nacionales de medición del empleo en el sector turístico.

El \textbf{Capítulo} \ref{fuentes-empleo} presenta las fuentes de información disponibles para la medición del empleo en ramas características del turismo a nivel provincial. Se describen los tipos de fuentes, sus principales características, ventajas y limitaciones. El capítulo se concentra en aquellas fuentes disponibles al público general, mostrando ejemplos de uso a partir de información reciente.

\hypertarget{medicion-actividad}{%
\chapter{\texorpdfstring{\textbf{La medición de la actividad económica en turismo}}{La medición de la actividad económica en turismo}}\label{medicion-actividad}}

Este capítulo aborda la temática de la medición del impacto económico del turismo en una región, distinguiendo entre las diferentes perspectivas posibles y resumiendo las principales recomendaciones internacionales y comentando el antecedente de la medición en Argentina y en otros países del mundo.
El turismo es un fenómeno social y cultural, pero también económico.
Dos circunstancias han incentivado profundamente el interés por el cálculo de la contribución económica del turismo:

\begin{itemize}
\item
  La profunda interrelación que genera entre varias actividades económicas, como el transporte, el alojamiento, la alimentación, los servicios culturales, entre otros, hacen del turismo una actividad transversal a la economía, capaz de motorizar varios sectores muy diversos al mismo tiempo.
\item
  Su creciente relevancia en las economías tanto en regiones subnacionales, como ser una provincia, como en naciones enteras.
\end{itemize}

Ambos eventos hicieron que el turismo cobrara relevancia dentro del mapa sectorial y político, por lo que una correcta medición de su contribución económica brinda elementos para la mejor toma de decisiones que contribuyan al bienestar económico de las sociedades.
La medición del impacto económico del turismo puede realizarse desde la perspectiva de la demanda\footnote{Se puede consultar el capítulo 2 del \href{https://www.e-unwto.org/doi/book/10.18111/9789213612392}{CST:RMC 2008}, para más detalle sobre la metodología de esta perspectiva.}, la perspectiva de la oferta o bien desde la conciliación entre ambas perspectivas, lo que se ha definido como Cuenta Satélite de Turismo.
La perspectiva de la demanda consiste en medir todos los gastos en bienes y servicios que realiza un visitante en el contexto de su viaje, sin importar si el sector económico que provee dichos bienes o servicios es característico del turismo o no (para mayor definición de este concepto ver la sección siguiente sobre ramas características del turismo).

En cuanto a la perspectiva de la oferta\footnote{Se puede consultar el capítulo 3 del \href{https://www.e-unwto.org/doi/book/10.18111/9789213612392}{CST:RMC 2008}, para más detalle sobre la metodología de esta perspectiva.}, la misma consiste en la medición de la producción realizada por todas aquellas industrias consideradas como características del turismo.
Es importante notar, en este punto, que el foco de esta perspectiva está puesto en la producción total de algunas industrias de la economía, sin tener en cuenta quiénes efectivamente consumen sus productos (ya sean bienes o servicios).
Es decir, la producción de estas industrias podría ser consumida por visitantes durante un viaje turístico o bien por personas que no están realizando un viaje.

Finalmente, se puede realizar una conciliación entre la demanda de los visitantes y la oferta de las industrias turísticas (y del total de la economía).
El centro de esta perspectiva se encuentra en poder calcular el valor agregado que se generó en una economía en particular para atender directamente a la demanda de bienes y servicios realizada por los visitantes y compararlo con el valor agregado del total.
La herramienta estadística apropiada para la realización de este tipo de medición es una Cuenta Satélite de Turismo.
Ésta reúne diferentes fuentes de información sobre oferta de la economía y demanda turística y las concilia a fin de obtener la proporción del valor agregado generado por la demanda turística sobre el valor agregado total de la economía.

Las recomendaciones metodológicas para la realización de una Cuenta Satélite de Turismo han sido extensamente desarrolladas por la Organización Mundial del Turismo (OMT).
La serie de recomendaciones más actualizada puede encontrarse en el documento publicado por la OMT titulado ``Cuenta satélite de turismo: Recomendaciones sobre el marco conceptual, 2008'' (\href{https://www.e-unwto.org/doi/book/10.18111/9789213612392}{CST:RMC 2008}).
El presente documento se enfocará en el desarrollo más profundo de la perspectiva de la oferta, para poder proveer una herramienta concreta para estimar la contribución del turismo en cada economía local.
Si bien una cuenta satélite de turismo resulta una medición más precisa del impacto económico del turismo, ya que involucra una estimación de la oferta consumida directamente por visitantes, requiere de un marco conceptual más desarrollado que provea no solo de información desagregada sobre la producción de las industrias sino también de información detallada sobre la canasta de consumo de los visitantes durante sus viajes en la región, y que, ambos conjuntos de información, sean consistentes entre sí.

AGREGAR CST:RMC 2008 A BIBLIOGRAFÍA

\hypertarget{ramas-de-actividad-econuxf3mica}{%
\section{Ramas de Actividad Económica}\label{ramas-de-actividad-econuxf3mica}}

La evaluación de un sector económico complejo como el turismo plantea el desafío de definir correctamente las actividades involucradas en el mismo y detectar toda la oferta existente.
Con el fin de realizar esta definición, se utilizará la Clasificación Industrial Internacional Uniforme (CIIU), cuya última versión disponible corresponde a la \href{https://unstats.un.org/unsd/publication/seriesm/seriesm_4rev4s.pdf}{revisión 4}, que constituye una estructura de clasificación coherente y consistente de las actividades económicas, basada en un conjunto de conceptos, definiciones, principios y normas de clasificación, de reconocimiento internacional.
La estructura de la CIIU es un formato estándar que permite organizar la información detallada sobre la situación de una economía de acuerdo con principios y percepciones económicas.
Tomando como base la \href{https://unstats.un.org/unsd/publication/seriesm/seriesm_4rev4s.pdf}{CIIU Rev.~4} (ONU, 2010), la Tabla 2 presenta las 21 categorías individuales de la CIIU.
En la misma, se resaltan las ramas que, total o parcialmente, son características de la industria del turismo:

\begin{figure}

{\centering \includegraphics[width=0.8\linewidth]{imagenes/figura1.1} 

}

\caption{Ramas de Actividad Económica según el Código Industrial Internacional Uniforme}\label{fig:ciiu}
\end{figure}

La CIIU abarca generalmente todas las actividades productivas, es decir, las actividades económicas comprendidas dentro de la frontera de producción del Sistema de Cuentas Nacionales (SCN).
Esas actividades económicas se subdividen en una estructura jerárquica integrada por cuatro niveles de categorías mutuamente excluyentes.
Las categorías del nivel superior de la clasificación se denominan secciones, identificadas por un código alfabético que tienen por objeto facilitar el análisis económico.
Dichas secciones se subdividen en las actividades productivas de grandes grupos, como ``Agricultura, ganadería, silvicultura y pesca'' (sección A), ``Industrias manufactureras'' (sección C) o ``Información y comunicaciones'' (sección J).
La clasificación se estructura a partir de esas secciones en categorías cada vez más detalladas, identificadas por un código numérico, que es de dos dígitos para las divisiones, de tres dígitos para los grupos, y de cuatro dígitos para las clases (el nivel más desagregado).
Cada establecimiento de la economía se clasificará en una rama de actividad específica según cuál sea su actividad principal, es decir, aquella que genere el mayor valor agregado.

\hypertarget{ramas-caracteruxedsticas-del-turismo}{%
\subsection{Ramas Características del Turismo}\label{ramas-caracteruxedsticas-del-turismo}}

Como se mencionó anteriormente la CIIU está construida sobre un marco conceptual basado en la oferta en el cual se agrupan las unidades de producción en ramas detalladas, priorizando similitudes de su actividad económica, teniendo en cuenta los insumos, los procesos y la tecnología de producción, las características de los productos y los usos a los que se destinan.
Para avanzar es importante definir algunos conceptos en cuanto a la clasificación de la oferta en característica del turismo o no.
Es posible identificar tres agrupaciones por el lado de la oferta que pueden clasificarse como características del turismo: industrias, actividades y productos.
La vinculación entre estas tres es la siguiente: Las industrias características del turismo son aquellas cuya actividad principal es característica del turismo.
Las actividades características del turismo son aquellas que producen principalmente productos característicos del turismo.
Según el párrafo 5.10 de las \href{https://unstats.un.org/unsd/publication/seriesm/seriesm_83rev1s.pdf}{RIET 2008}: "Los productos característicos del turismo son aquellos que cumplen uno o ambos de los siguientes criterios:

\emph{a. El gasto turístico en el producto debería representar una parte importante del gasto total turístico (condición de la proporción que corresponde al gasto/demanda)}

\emph{b. El gasto turístico en el producto debería representar una parte importante de la oferta del producto en la economía (condición de la proporción que corresponde a la oferta). Este criterio supone que la oferta de un producto característico del turismo se reduciría considerablemente si no hubiera visitantes".}

En adelante, el foco estará puesto en las industrias características del turismo ya que son la base del análisis de la contribución económica del turismo desde la perspectiva de la oferta.
En conclusión, el manual \href{https://www.e-unwto.org/doi/book/10.18111/9789213612392}{CST:RMC 2008} propone la consideración de 10 ramas (industrias) turísticas para la comparación internacional. Adicionalmente, se incorporan 2 líneas de industrias que pueden ser adicionadas por cada región si lo creyera pertinente, según el conocimiento de cómo se desarrolla el turismo dentro de sus límites geográficos. A continuación se presentan estas ramas con el detalle de cada código CIIU (Rev.4) asociado:

\begin{figure}

{\centering \includegraphics[width=0.8\linewidth]{imagenes/figura1.2} 

}

\caption{Industrias turísticas y sus códigos CIIU Rev. 4.}\label{fig:ciiu2}
\end{figure}

Se debe tener en cuenta que dichas RCT se consideran como una propuesta, sujeta a que la composición del consumo turístico difiere según la región geográfica. No obstante, es prioritario seguir la agrupación según los códigos CIIU con el fin de mantener la comparabilidad a nivel internacional.

\hypertarget{consideraciones-sobre-el-alcance}{%
\subsection{Consideraciones sobre el alcance}\label{consideraciones-sobre-el-alcance}}

Como se mencionó en el apartado anterior, las industrias turísticas son aquellas cuya actividad principal es característica del turismo. La producción que ellas realicen será el centro de la perspectiva de la oferta para la medición de la contribución económica del turismo. Este enfoque tiene ciertos límites a su alcance que son importantes de explicitar.

\textbf{Destinatario final de la producción}

La perspectiva de la oferta realiza una medición de la contribución del turismo desde el lado de la producción de las industrias turísticas, sin tener en consideración quién consume dicha producción, ni si el mismo es visitante o no.
Por lo tanto, dentro de esta perspectiva está incluida la producción que las industrias turísticas realizan pero que no es consumida por visitantes, sino por residentes. En actividades como alojamiento esto no resulta demasiado importante, ya que la mayor parte de la producción de servicios de alojamiento es consumida por visitantes, pero en la actividad de bares y restaurantes o en los servicios de esparcimiento, etc., resulta evidente la importancia de la participación de los residentes en el consumo total.
La situación inversa se presenta en el caso de los consumos de bienes o servicios que los visitantes podrían hacer de industrias no clasificadas como características del turismo. Estos consumos no estarán incluidos en la estimación de la contribución económica del turismo desde la perspectiva de la oferta. Este es el caso del comercio minorista, por ejemplo. Este no es catalogado como una rama característica del turismo aunque los visitantes realizan cuantiosos consumos en esta rama (vestimenta, souvenirs, alimentos y bebidas sin preparación, etc).

\textbf{Establecimientos y actividad principal}

Cada industria turística estará compuesta por todos aquellos establecimientos cuya actividad principal es una actividad característica del turismo. Por lo tanto, si un establecimiento realiza actividades turísticas solo de manera secundaria, no estará incluido dentro de la industria turística caracterizada por esa actividad, por lo tanto, quedará fuera del cálculo de impacto económico del turismo. Un ejemplo de esta situación podría ser: un viñedo obtiene el mayor valor agregado de su actividad de cultivo de vides (actividad principal), pero, a su vez, podría tener alguna casa de huéspedes que ofrezca a visitantes para hospedarse (actividad secundaria). La oferta de esta actividad secundaria de alojamiento turístico no estará incluida en la producción de las industrias turísticas, simplemente porque el establecimiento estará clasificado en la industria asociada a la agricultura, y la misma no es característica del turismo.
Contrariamente, la producción de las industrias turísticas sí incluirá la producción de actividades que los establecimientos turísticos presten de manera secundaria y que no estén vinculadas al turismo. Un ejemplo de este caso podría ser: una agencia de viajes obtiene su mayor valor agregado a partir de la comercialización de paquetes y servicios turísticos en general (actividad principal), pero adicionalmente presta un servicio de concesión de créditos para la financiación de los viajes turísticos, por el cual cobra un interés (actividad secundaria).
En conclusión, es importante notar que la perspectiva de la oferta consiste en un recorte de la producción de una región, siguiendo las recomendaciones internacionales para realizarlo, pero no incorpora en el análisis la demanda efectivamente realizada por visitantes, por lo que no provee una medida de la producción que exclusivamente se generó para atender a los visitantes\footnote{Como se mencionó previamente, el objetivo de este trabajo es profundizar sobre la perspectiva de la oferta para la medición del impacto económico del turismo en una región. Esto implica discriminar con la mayor precisión posible las ramas características, lo cual supone lo ``máximo'' esperable desde el enfoque de la oferta o ramas características.}.

\hypertarget{dificultades-derivadas-del-nivel-de-desagregaciuxf3n-de-las-ramas}{%
\subsection{Dificultades derivadas del nivel de desagregación de las ramas}\label{dificultades-derivadas-del-nivel-de-desagregaciuxf3n-de-las-ramas}}

De acuerdo a la descripción realizada sobre el formato de la CIIU, es importante notar que cuanto menor sea el nivel de apertura de la información por rama menor será la precisión de los resultados obtenidos. Por ejemplo, si una determinada fuente brinda los datos globales de Servicios de Transporte Terrestre, la estimación incluirá componentes no relacionados con el turismo (transporte de Carga, por ejemplo), lo cual implica una sobreestimación de la producción de las ramas características del sector.
En contrapartida, cuando la información se presenta en un nivel de detalle mayor permite discriminar mejor qué es y qué no es característico del sector bajo estudio. Por ejemplo, si el transporte terrestre permite distinguir el transporte urbano de pasajeros, del transporte interurbano de pasajeros y del transporte de cargas, resulta evidente que el único que debe ser seleccionado como actividad característica del turismo es el transporte interurbano de pasajeros.

\hypertarget{mediciuxf3n-de-la-oferta-de-las-industrias-turuxedsticas}{%
\section{Medición de la oferta de las industrias turísticas}\label{mediciuxf3n-de-la-oferta-de-las-industrias-turuxedsticas}}

Para una correcta medición del impacto económico del turismo desde la perspectiva de la oferta es fundamental contar con estadísticas robustas sobre la producción de las industrias turísticas. En este apartado se buscará realizar algunas recomendaciones para lograr este objetivo.

\hypertarget{recomendaciones-generales-sobre-la-mediciuxf3n}{%
\subsection{Recomendaciones generales sobre la medición}\label{recomendaciones-generales-sobre-la-mediciuxf3n}}

En primer lugar, se deberá contar con encuestas a las industrias, o bien registros administrativos, que permitan una estimación fiable de la producción en un período determinado. Por lo general, las encuestas anuales a los establecimientos facilitan la obtención de información sobre sus actividades, su producción y sus costos intermedios (para poder calcular el valor agregado de cada uno).
En segundo lugar, como se mencionó en el apartado 1.1.3, la posibilidad de contar con información desagregada de la producción de las RCT es clave porque estará directamente vinculada con la precisión de la estimación de impacto económico. Por lo tanto, se recomienda buscar la generación de estadísticas de oferta que provean una desagregación de las ramas a un nivel de cuatro dígitos de la CIIU, siempre que sea posible.
En tercer lugar, será importante considerar a la informalidad, dependiendo el caso de cada región y de cada actividad a estimar. En este caso, las encuestas a hogares suelen ser la herramienta más útil para calcular la informalidad en la prestación de servicios (particularmente en sectores no tan regulados como el alojamiento, los restaurantes o los servicios de excursiones, por ejemplo).

\hypertarget{el-caso-particular-de-las-agencias-de-viaje-y-los-operadores-turuxedsticos}{%
\subsection{El caso particular de las agencias de viaje y los operadores turísticos}\label{el-caso-particular-de-las-agencias-de-viaje-y-los-operadores-turuxedsticos}}

Los operadores turísticos combinan dos o más servicios turísticos (transporte, alojamiento, excursiones, etc) y los comercializan a los visitantes de manera directa o a través de agencias de viaje. Las agencias de viaje tratan directamente con los clientes minoristas y proveen el servicio de organización de viajes.
En el caso de este tipo de establecimientos, es importante asegurar que la información recopilada permite discriminar los cobros de las agencias y operadores en sus dos partes fundamentales: 1) el valor de los servicios vendidos (que fueron adquiridos por las agencias u operadores directamente a los prestadores de cada servicio turístico en particular) y 2) el margen de comercialización (el pago neto que reciben agencias y operadores por el servicio de organización prestado).
Con el fin de obtener información acerca de los márgenes de comercialización incluidos en los servicios de organización prestados, podrían realizarse encuestas particulares a este tipo de establecimiento que indaguen directamente sobre sus estructuras de costos y los márgenes de ganancia que adicionan al momento de la venta.

\hypertarget{indicadores-macroeconuxf3micos-por-el-lado-de-la-oferta-turuxedstica}{%
\section{Indicadores macroeconómicos por el lado de la oferta turística}\label{indicadores-macroeconuxf3micos-por-el-lado-de-la-oferta-turuxedstica}}

Con el objetivo de caracterizar a la oferta turística de una región, se recomienda el cálculo de dos indicadores agregados: el \textbf{Valor Bruto de Producción de las Industrias Turísticas (VBPIT)} y el \textbf{Valor Agregado Bruto de las Industrias Turísticas (VABIT)}.

\hypertarget{valor-bruto-de-producciuxf3n-de-las-industrias-turuxedsticas-vbpit}{%
\subsection{Valor Bruto de Producción de las Industrias Turísticas (VBPIT)}\label{valor-bruto-de-producciuxf3n-de-las-industrias-turuxedsticas-vbpit}}

Según las cuentas nacionales, el VBP es el valor total de los bienes y servicios producidos en un territorio económico durante un determinado período de tiempo por los agentes económicos residentes o unidades económicas (establecimientos).
Por lo tanto, el VBP de las industrias turísticas (VBPIT) será el valor de todos los bienes y servicios producidos por las RCT en un territorio y en un período dados.
La valuación del VBPIT se realiza a precios básicos. Los precios básicos son los precios antes de sumar los impuestos sobre los productos y de restar las subvenciones sobre los productos (SCN 2008). Es decir, en esta valuación no están incluidos impuestos como IIBB, IVA, impuestos específicos, a los créditos y débitos bancarios, entre otros.
Se recomienda la generación de este indicador para un período de tiempo de un año, ya que permite neutralizar la estacionalidad propia de la actividad turística y reduce el costo de la obtención de información para un período de tiempo menor.
Además del valor monetario del VBPIT, se recomienda calcular su proporción en el total del VBP de la economía bajo estudio.

\hypertarget{valor-agregado-bruto-de-las-industrias-turuxedsticas-vabit}{%
\subsection{Valor Agregado Bruto de las Industrias Turísticas (VABIT)}\label{valor-agregado-bruto-de-las-industrias-turuxedsticas-vabit}}

El VAB es un agregado macroeconómico que mide la producción nueva generada por una industria, país o región, es decir, solo considera el valor que agrega cada una de las industrias a la economía en su totalidad, sin duplicar producciones. Para esto se parte del VBP y a éste se le restan los consumos intermedios (insumos utilizados en el proceso de producción del bien o servicio en cuestión).
El VAB de las industrias turísticas (VABIT) será el valor agregado de todas los establecimientos pertenecientes a industrias consideradas como RCT, sin importar a quiénes estuvo orientada su producción: si a visitantes o a residentes. Como se mencionó en el apartado 1.1.2, en este enfoque, el valor agregado bruto no incluye el valor generado por otras industrias no turísticas cuyos productos hayan sido efectivamente adquiridos por visitantes.
Así como el VBPIT, el VABIT se mide a precios básicos.
Así como se sugirió para el VBPIT, se recomienda el cálculo del VABIT anual.
Finalmente, también se recomienda realizar el cálculo de la proporción de VAB generado por las industrias turísticas en el total del VAB de la economía, para tener una noción del peso de las mismas en la economía general.

\hypertarget{antecedentes-en-argentina}{%
\section{Antecedentes en Argentina}\label{antecedentes-en-argentina}}

El Ministerio de Turismo y Deportes de la Nación realizó una estimación de la contribución económica del turismo en la Argentina con base en el año 2004\footnote{El año 2004, es el único año para el cual las cuentas nacionales de Argentina cuentan con Cuadros de Oferta-Utilización (COU). Estos cuadros consisten en un esquema estadístico, elaborado para el año 2004 por la Dirección de Cuentas Nacionales (DNCN) del INDEC, que cuenta con:
  un cuadro de oferta: tiene a las actividades económicas en sus columnas (clasificadas con la CLANAE 2004, compatible con la CIIU Revisión 3) y a los productos (bienes o servicios) producidos por cada una de ellas en sus filas (clasificados con la Clasificación Central de Productos (CPC) de las Naciones Unidas, Revisión 1.1). Presenta entonces la producción de cada actividad económica a nivel de producto, valuados en pesos y a precios básicos; y
  un cuadro de utilización: tiene a las actividades económicas y a los sectores de demanda final (los hogares, el gobierno, el resto del mundo y las empresas como formadoras de capital bruto) en sus columnas, mientras que en las filas se encuentran los productos (bienes o servicios) utilizados por cada una de las columnas (consumo intermedio, en el caso de las actividades económicas, y consumo final en el caso de los sectores de demanda final). Los valores son presentados en pesos, pero a precios de comprador, es decir, se adicionan impuestos, márgenes de comercio y transporte y se restan las subvenciones.} y para los años 2016-2019, con la colaboración de la Dirección Nacional de Cuentas Nacionales (DNCN) del INDEC.
La DNCN proveyó, a pedido del Ministerio de Turismo y Deportes, información de la producción de las actividades económicas a un nivel de desagregación de 4 dígitos de la CIIU Rev.~3, y en algunos casos a 5 o 6 dígitos. De esta manera, fue posible la estimación del impacto económico del turismo desde la perspectiva de la oferta.
Dada la posibilidad de incluir bienes y/o servicios característicos específicos de cada país, para la CST-A se determinó que se considerarían como característicos del turismo a los servicios de expendio minorista de combustibles para automotores (estaciones de servicio). Por ende, su actividad relacionada, venta al por menor de combustible para automotores, también se consideró como característica de turismo.
Las \href{https://unstats.un.org/unsd/publication/seriesm/seriesm_83rev1s.pdf}{RIET 2008} mencionan que ``El combustible para los vehículos de motor (o para embarcaciones en los países insulares) podría representar asimismo un gasto importante en bienes en los países''. A su vez, se verificó que el gasto en combustibles representó un 10\% del gasto turístico interno que se calculó para la CST-A en el año 2004 para los turistas y un 30\% para los excursionistas.

De esta manera, la apertura de actividades características del turismo para la CST-A quedó determinada de la siguiente manera:

\begin{figure}

{\centering \includegraphics[width=0.8\linewidth]{imagenes/figura1.3} 

}

\caption{Actividades Características del Turismo en la CST-A}\label{fig:activcst}
\end{figure}

Por su lado, la producción y el valor agregado de cada industria turística quedó conformado de la siguiente manera:

\emph{En millones de pesos corrientes. A precios básicos.}

\begin{figure}

{\centering \includegraphics[width=0.8\linewidth]{imagenes/figura1.4} 

}

\caption{Cuentas de producción de las RCT, Argentina, 2004, 2016-2019}\label{fig:cst}
\end{figure}

Se puede observar que, en el año 2004 y para el total de la Argentina, el VBPIT fue de \$40.023,93 millones, mientras que el VABIT fue de \$19.111,52 millones. En términos relativos el VBPIT fue un 4,8\% del VBP del total de la economía, mientras que el VABIT representó un 4,6\% del VAB total. Hacia los años de la serie 2016-2019 se observa un crecimiento de la producción y valor agregado de las industrias turísticas, dado que en alcanzaron valores más cercanos al 6\% y 5\%, respectivamente.

\hypertarget{medicion-empleo}{%
\chapter{\texorpdfstring{\textbf{La medición del empleo en turismo}}{La medición del empleo en turismo}}\label{medicion-empleo}}

Este capítulo propone un recorrido por los principales conceptos sobre la economía laboral, las metodologías de medición del empleo en el sector turístico (según las recomendaciones de la OMT) y su aplicación tanto en Argentina como en el resto del mundo.
El mercado laboral es definido como el mercado en donde confluyen la demanda y la oferta de trabajo. Los componentes del mercado son expresados en cantidades (puestos de trabajo y personas ocupadas) y precios (salarios) que generalmente se asocian al bienestar de la economía.
La demanda en el mercado de trabajo es efectuada por las empresas para poder desempeñar su actividad económica. La cantidad de trabajadores a emplear dependerá de la cantidad de puestos de trabajo que se requiera ocupar para la actividad en cuestión.
En tanto que la oferta de trabajo es efectuada por las personas a partir de ofrecer sus servicios a las empresas, o trabajar en forma independiente, a cambio de una retribución.

La Figura \textbf{NUMERO} representa una perspectiva esquemática del mercado de trabajo con sus componentes (elementos estructurales) y relaciones. Muestra que las ``personas'' representan el lado de la oferta en el mercado de trabajo, mientras que los ``puestos'' el lado de la demanda. Los casilleros cuadrados en las primeras dos filas incluyen a los empleadores y a los hogares, además de los puestos y las personas. Los ``puestos'' y las ``personas'' se vinculan a través de los empleos.

Para mayor información consulte el \href{https://webunwto.s3-eu-west-1.amazonaws.com/imported_images/25994/ilo_b_sp.pdf}{documento de la OIT}

En la tercera fila aparecen más casilleros que muestran las subcategorías de puestos y personas ocupadas. El sistema de contabilidad del trabajo requiere que se establezcan estimaciones para todos sus componentes y las relaciones de éstos.

Como se explicará en las secciones siguientes, el empleo en turismo no puede ser observado en forma directa, al menos aquel que se refiere al empleo estrictamente relacionado con los bienes y servicios adquiridos por los visitantes, sin importar si la industria que los produjo es turística o no, por lo tanto, este documento analiza el enfoque de oferta (ó ramas características del turismo\footnote{Se utilizarán indistintamente los términos ``Industria'' o ``Rama''})

\begin{figure}

{\centering \includegraphics[width=0.8\linewidth]{imagenes/figura2.1} 

}

\caption{Marco conceptual del Mercado de Trabajo}\label{fig:empleooit}
\end{figure}

\hypertarget{definiciones-generales-sobre-mediciuxf3n-del-empleo}{%
\section{Definiciones generales sobre medición del empleo}\label{definiciones-generales-sobre-mediciuxf3n-del-empleo}}

Es importante conocer a qué se refieren las estadísticas laborales cuando miden trabajo y empleo. Partiendo de la definición de la \href{https://www.ilo.org/global/lang--es/index.htm}{OIT}, el \textbf{trabajo}\footnote{En la actualidad, siguiendo las recomendaciones de la OIT, el concepto trabajo ha sido ampliado a ``Trabajo Decente''. Dicho concepto reconoce que el trabajo promueve la dignidad personal, el crecimiento económico, la estabilidad familiar, la paz en la comunidad, la democracia, la productividad y el desarrollo de las empresas.} corresponde al conjunto de actividades humanas, remuneradas o no, que producen bienes o servicios en una economía, o que satisfacen las necesidades de una comunidad o proveen los medios de sustento necesarios para los individuos. En este contexto, el \textbf{empleo} es definido como el trabajo efectuado a cambio de una remuneración que puede ser denominado salario, sueldo, comisión, propina, pago a destajo o pago en especie sin importar la categoría de empleo (patrón, asalariado, cuentapropista, etc.).

Por otra parte, la OIT categoriza un empleo según el tipo de contrato de trabajo explícito o implícito del titular con otras personas u organizaciones. A partir del año 1993 se creó la Clasificación Internacional de la Situación en el Empleo (CISE, 1993)\footnote{\href{http://www.ilo.org/public/spanish/bureau/stat/download/res/icse.pdf}{Resolución CISE, 1993}} que sirve para agrupar las distintas modalidades vigentes y para comprender su alcance según el sector que se pretenda estudiar. El criterio básico utilizado para definir cada grupo de categoría o clasificación ocupacional está basado en el tipo de \textbf{riesgo económico}, un elemento del cual depende exclusivamente la solidez del vínculo entre la persona y el empleo, y el tipo de responsabilidad que tiene el trabajador sobre el establecimiento.

Las personas con \textbf{\emph{empleos independientes}} pueden dividirse en dos grupos, aquellas que tienen trabajadores asalariados a cargo se clasifican como \textbf{\emph{``empleadores''}} o \textbf{\emph{``patrones''}}, mientras que aquellas que no los tienen se clasifican como \textbf{\emph{``trabajadores por cuenta propia''}}. El término empleo independiente alcanza aquel tipo de trabajo donde la remuneración depende directamente de los beneficios (o del potencial para realizar beneficios) derivados de los bienes o servicios producidos (el consumo propio forma parte de los beneficios). El titular toma las decisiones operacionales que afectan a la empresa, o delega tales decisiones, pero mantiene la responsabilidad por el bienestar de la empresa.

Por otro lado, los trabajadores con un empleo asalariado son clasificados dentro de la categoría \textbf{\emph{``empleo en relación de dependencia''}}, situación que refiere a aquel trabajador que recibe una remuneración básica que no depende directamente de los ingresos de la unidad para la que trabaja.

Cabe aclarar que todas las categorías mencionadas anteriormente pueden identificarse tanto en el marco de la economía formal como de la economía informal. El término \textbf{\emph{``economía informal''}}\footnote{\href{https://www.ilo.org/wcmsp5/groups/public/---ed_norm/---relconf/documents/meetingdocument/wcms_218350.pdf}{OIT: ``La transición de la economía informal a la economía formal'', Conferencia Internacional del Trabajo, 103.ª reunión, 2014}} hace referencia \emph{``al conjunto de actividades económicas desarrolladas por los trabajadores y las unidades económicas que, tanto en la legislación como en la práctica, están insuficientemente contempladas por sistemas formales o no lo están en absoluto. Las actividades de esas personas o empresas pueden no estar alcanzadas por la ley, es decir, se desempeñan al margen de ella; o bien no están contempladas en la práctica, es decir que, si bien estas personas operan dentro del ámbito de la ley, ésta no se aplica o no se cumple''} (OIT -- Conferencia Internacional de Trabajo, 2014).

Además, existen grandes diferencias entre los trabajadores de la economía informal en cuanto a ingresos (nivel, regularidad, estacionalidad), situación en el empleo (asalariados, empleadores, trabajadores por cuenta propia, trabajadores ocasionales, trabajadores domésticos), sector (comercio, agricultura, industria), tipo de empresas y tamaño de las mismas, ubicación geográfica (medio urbano o rural), protección social (contribuciones a la seguridad social), y protección del empleo (tipo y duración del contrato, derecho a vacaciones anuales).
Por lo tanto, en un sistema económico los trabajadores pueden desempeñarse bajo una modalidad de Empleo Formal o Empleo Informal. La distinción entre uno u otro está dada por el cumplimiento de la legislación laboral en relación con las personas involucradas en la unidad productiva o en su propio emprendimiento económico cumpliendo con alguna de las normas que regulan sus actividades económicas y las obligaciones previsionales.

En síntesis, se considera que los trabajadores se encuentran en la modalidad de un \textbf{\emph{Empleo Informal}}\footnote{Según la OIT, las razones pueden ser las siguientes: la no declaración de los empleos o de los asalariados; empleos ocasionales o empleos de limitada o corta duración; empleos con un horario o un salario inferior a un límite especificado (por ejemplo para cotizar a la seguridad social); el empleador es una empresa no constituida en sociedad o una persona miembro de un hogar; el lugar de trabajo del asalariado se encuentra fuera de los locales de la empresa del empleador (por ejemplo, los trabajadores fuera del establecimiento y sin contratos de trabajo); empleos a los cuales el reglamento laboral no se aplica, no se hace cumplir o no se hace respetar por otro motivo.} si su relación de trabajo, de derecho o de hecho, no está sujeta a la legislación laboral nacional, la protección social o a determinadas prestaciones relacionadas con el empleo (cobertura jubilatoria con descuento, cobertura jubilatoria con aporte voluntario, preaviso al despido, indemnización por despido, vacaciones anuales pagadas o licencia pagada por enfermedad, etc.).

Por lo tanto, las principales definiciones a tener en cuenta son (OIT, 2004)REFERENCIA:

\begin{itemize}
\item
  \textbf{Persona ocupada\footnote{En Argentina se considera como persona ocupada a todos los individuos que tengan cierta edad específica (10 años o más) y que durante un período de referencia (una semana) hayan trabajado al menos una hora. Incluye: a) las personas que durante el período de referencia realizaron algún trabajo de al menos una hora, sin importar si recibieron pago (en dinero o en especie) o no por dicha actividad; b) las personas que tienen una ocupación, pero que no están trabajando temporalmente durante el período de referencia y mantienen un vínculo formal con su empleo. Integran este grupo los ocupados que no trabajaron en la semana, por vacaciones, licencia por enfermedad u otros tipos de licencias, suspendidos con pago y ausentes por otras causas laborales (mal tiempo, averías mecánicas, escasez de materias primas, etc.) con límite de tiempo de retorno. Se incluyen también dentro de esta categoría a las personas que tienen un negocio o empresa y no trabajaron por causas circunstanciales durante el período de referencia.}:} es un individuo físico que realiza ciertas tareas laborales, pudiendo ocupar uno o más puestos de trabajo;
\item
  \textbf{Puesto ocupado:} corresponde a una persona empleada para realizar un conjunto de tareas en una empresa o negocio;
\item
  \textbf{Empleador o Patrón:} es aquel que, trabajando por su cuenta o con uno o más socios, ha contratado a una o a varias personas para que trabajen en su empresa como ``asalariados'' a lo largo de un período continuo que incluye el período de referencia. Los titulares pueden tomar las decisiones operacionales que afectan a la empresa, o bien delegarlas, pero mantienen la responsabilidad por el bienestar de la empresa;
\item
  \textbf{Trabajador por cuenta propia:} es aquel trabajador que, trabajando por su cuenta o con uno o más socios, no ha contratado a ningún ``asalariado'' de manera continua durante el período de referencia. Cabe notar que durante el período de referencia los miembros de este grupo pueden haber contratado ``asalariados'', siempre y cuando lo hagan de manera esporádica;
\item
  \textbf{Asalariado:} es aquel trabajador que tiene un empleo en relación de dependencia y que posee, por lo tanto, un contrato de trabajo implícito o explícito (oral o escrito) con el mismo empleador de manera continua. Los empleados asalariados son los trabajadores que reciben una remuneración básica, la cual no depende directamente de los ingresos de la unidad para la que trabaja. Además, la organización empleadora es responsable por el pago de las cargas fiscales y de las contribuciones de la seguridad social del empleado (a partir de las exigencias de la legislación nacional de trabajo).
\end{itemize}

\textbf{AGREGAR ALGUNOS CONCEPTOS DE:}
\textbf{EMPLEO VERDE }
\textbf{EMPLEO DECENTE}

\hypertarget{recomendaciones-internacionales-sobre-el-empleo-en-industrias-turuxedsticas}{%
\section{Recomendaciones Internacionales sobre el empleo en industrias turísticas}\label{recomendaciones-internacionales-sobre-el-empleo-en-industrias-turuxedsticas}}

Las Recomendaciones Internacionales sobre Estadísticas de Turismo (\href{https://unstats.un.org/unsd/publication/seriesm/seriesm_83rev1s.pdf}{RIET 2008}) describen dos formas de medir el empleo relacionado con el turismo. Por un lado, se denomina Empleo Turístico \emph{``al empleo estrictamente relacionado con los bienes y servicios (característicos del turismo, el turismo conexo y otros) adquiridos por los visitantes y producidos por cualquiera de las industrias del turismo y otras industrias''}, mientras que el Empleo en las Industrias Turísticas se refiere al empleo en las actividades características del turismo.
Las RIET 2008 sugieren un marco metodológico para medir el nivel y características del empleo generado por la industria del turismo \emph{desde una perspectiva de la oferta}, a través de la selección de empresas o industrias características del turismo. Es decir, se tiene en cuenta el empleo generado en una selección de ramas de actividad económica características del Turismo.
Además, menciona las particularidades del sector a la hora de estimar el empleo turístico, haciendo referencia a que las actividades características del turismo suelen requerir abundante mano de obra, que, si bien puede asociarse a la producción total de un establecimiento, no puede asignarse a una producción particular \emph{sin la utilización de hipótesis y de procedimientos de modelización}.
Por este motivo, no se puede observar directamente el empleo en turismo, haciendo referencia al empleo estrictamente relacionado con los bienes y servicios (característicos del turismo, conexos al turismo y de otro tipo) adquiridos por los visitantes y producidos por las industrias turísticas u otras industrias.
Dado que el objetivo de este trabajo es explorar las estadísticas del empleo en el sector turístico, el enfoque será puesto en el empleo en las industrias turísticas, es decir, el empleo en el sector, con independencia de que los productos y/o servicios fueran adquiridos por los turistas o no. Del mismo modo, no serán contempladas aquellas ramas de actividad que producen bienes o servicios que los visitantes eventualmente pueden consumir, pero que no constituyen industrias típicas del sector.

\hypertarget{empleo-en-las-ramas-de-actividad-econuxf3mica}{%
\section{Empleo en las ramas de actividad económica}\label{empleo-en-las-ramas-de-actividad-econuxf3mica}}

Las estimaciones del empleo en el sector se verían resueltas con facilidad si pudieran ser calculadas en base al volumen de los bienes y servicios consumidos únicamente por los turistas. Sin embargo, la asociación de un nivel de empleo a un volumen de bienes y servicios es muy difícil de lograr y de justificar teóricamente. Por esta razón, para identificar el empleo en las industrias turísticas se debe recurrir a la selección de las ramas características, lo cual, además, permite identificar la composición del empleo turístico por categoría ocupacional.
Con este objetivo, se utilizará la \href{https://unstats.un.org/unsd/publication/seriesm/seriesm_4rev4s.pdf}{CIIU Rev.~4} y la correspondiente delimitación de ramas características del turismo, tal como se explicita en la sección 1.1.1 del presente documento.

\hypertarget{dificultades-generales-para-la-estimaciuxf3n-del-empleo-en-la-industria-turuxedstica}{%
\section{Dificultades generales para la estimación del empleo en la industria turística}\label{dificultades-generales-para-la-estimaciuxf3n-del-empleo-en-la-industria-turuxedstica}}

Desde los postulados planteados, el análisis de las fuentes que brindan información sobre el mercado de trabajo en Argentina permite adelantar un conjunto de problemas específicos para el abordaje de la medición del empleo en turismo a partir del enfoque de ramas características. La metodología de estimación diseñada procura solucionar, de la \emph{mejor forma posible}, los siguientes problemas:

\begin{itemize}
\tightlist
\item
  \textbf{Identificación de las empresas del sector:} dificultad para definir con exactitud las empresas (y sus trabajadores) dedicadas a ofrecer productos y/o servicios al sector y para extraer aquellos componentes no turísticos a partir del cruce de las distintas fuentes de información, debido a los diferentes niveles de desagregación con que se presenta la información y/o a la utilización de distintos nomencladores de la actividad económica.
\item
  \textbf{Estacionalidad:} las fluctuaciones de la demanda turística introducen variaciones en el volumen del empleo. Esto es especialmente claro en el dominio de la pequeña empresa familiar, en donde la contratación temporal permite cubrir épocas de mayor demanda. Adela Mariscal (2005) en su trabajo ``Mercado de Trabajo y turismo en Andalucía'' señala que donde existen estructuras consolidadas (pymes, cadenas hoteleras, etc.), la estabilidad y la calidad del empleo son mayores, pero que en los casos de micropymes, cooperativas, sociedades anónimas laborales, etc., y donde además puede existir vulnerabilidad económica y territorial, el empleo es altamente estacional. Esta reflexión, sin dudas, aplica también para el caso argentino.
\item
  \textbf{Trabajo informal:} el alto nivel de empleo informal puede implicar la subestimación del impacto del turismo en el empleo. Según el documento ``Empleo y Recursos Humanos en España'' (Sancho, 1998), los principales sectores de la economía informal son la agricultura y los servicios (hoteles y restaurantes, servicios privados de limpieza y trabajo doméstico). Además, existe una alta rotación de los empleos, que es más significativa en el sector hotelería y restaurantes, con tasas superiores al 30\% (Greffe, 1994; OIT, 1997). Los principales afectados son los asalariados de pequeños y medianos establecimientos, los cuentapropistas y los trabajadores familiares no remunerados. Resulta evidente que obtener información robusta sobre este subuniverso es más complejo que hacerlo sobre los establecimientos o empleados formalizados.
\item
  \textbf{Periodicidad:} continuidad en la disponibilidad de los datos, así como antigüedad de las series.
\item
  \textbf{Cobertura:} falta de cobertura de la totalidad del territorio (geográfico o económico) bajo estudio.
\item
  \textbf{Robustez y fiabilidad:} las encuestas por muestreo implican diseños que no permiten realizar estimaciones pequeñas con márgenes de error razonables. Esto implica necesariamente el desarrollo de estrategias que utilicen las estimaciones del modo más agregado posible pero que, a su vez, permita dar cuenta de las aperturas relevantes que se procuran estimar (provincia, sector, categoría ocupacional).
\end{itemize}

** Se puede agregar una sección sobre recomendaciones generales a tener en cuenta para estimar el empleo. Tipo, qué sería bueno que tengan en cuenta al momento de relevar información como para poder obtener información robusta (símil sección 1.2.1 del cap 1)**

\hypertarget{antecedentes-internacionales-de-la-mediciuxf3n-del-empleo-en-turismo}{%
\section{Antecedentes internacionales de la medición del empleo en turismo}\label{antecedentes-internacionales-de-la-mediciuxf3n-del-empleo-en-turismo}}

Todo país que cuente con la CST podrá abordar el cálculo de empleo en el sector turístico con mayor facilidad, ya que la CST aporta información precisa al momento de estimar el número de personas ocupadas y puestos de trabajo creados por el turismo dentro de la economía del país. Algunos de los países que han desarrollado este sistema son Austria, Australia, Canadá, España, Francia, Nueva Zelanda y Estados Unidos.
La medición del empleo en el caso de Austria se inició en el año 2003. En la misma se incluye a todas las Industrias Turísticas definidas por las recomendaciones internacionales de la OMT para la CST, siendo el propósito final obtener información sobre el empleo en la industria del turismo a partir de la integración de las estadísticas laborales, conociendo a los empleos directos e indirectos de la industria del turismo. La estimación del modelo se realiza en base a los datos laborales existentes y la CST con la aplicación del marco Metodológico Recomendado 2008 (OMT, la OCDE, EUROSTAT). Para la estimación se utilizan varias fuentes de información. Desde el lado de la Demanda, acuden a las encuestas a hogares, mientras que desde la Oferta, utilizan estudios de las actividades económicas (por ejemplo, estadísticas de las empresas). Luego integran estadísticas basadas en registros administrativos (por ejemplo, archivos de seguridad social, reportes de impuestos o informes de empleo).
En el caso de España se utiliza una metodología de estimación del empleo a través de la matriz simétrica de insumo producto. El registro de datos para la estimación del empleo es, desde el año 1999, de base anual. Es una técnica basada en la oferta anual, donde convergen varias fuentes de información estadísticas del empleo, como por ejemplo la Encuesta Anual de Servicios o la Encuesta de Población Activa, entre otras. El cálculo del Empleo Total se realiza a partir de la suma del Empleo Directo y el Empleo Indirecto en la Industria del Turismo. Esto implica suponer que el contenido en el empleo por unidad de producto es el mismo, tanto para la producción destinada al turismo como para el resto. El empleo directo se estima por la demanda turística, asignando una relación proporcional de la producción turística para cada rama de actividad. En tanto que el empleo indirecto se cuantifica con el vector de empleo por rama de actividad o por producto. Por tanto, contar con la CST permite aplicar ratios de empleo en el sector turístico para calcular volúmenes.
Como puede notarse, la baja cantidad de países con información robusta sobre el empleo denota en parte que el desarrollo de un sistema estadístico del sector turístico es incipiente en el mundo. Aún existe una gran necesidad de información sobre el creciente papel del turismo en la economía, es decir, sobre el impacto económico de la Industria Turística. De ahí la importancia de reunir datos fiables sobre la magnitud del turismo utilizando los mismos conceptos, definiciones y enfoques de medición para garantizar la comparabilidad internacional. La breve descripción de la experiencia de los países mencionados en los párrafos precedentes da cuenta que muchos ya han logrado armonizar su sistema de información en la CST, permitiendo así que gobiernos, empresarios y ciudadanos estén mejor preparados para diseñar e implementar políticas públicas y estrategias empresariales.

\hypertarget{antecedentes-de-la-mediciuxf3n-del-empleo-en-turismo-en-argentina}{%
\section{Antecedentes de la medición del empleo en turismo en Argentina}\label{antecedentes-de-la-mediciuxf3n-del-empleo-en-turismo-en-argentina}}

Para el caso de Argentina, existen algunos trabajos de investigación que tratan la estimación del empleo en el sector turístico. Uno de ellos es \emph{``El empleo en ramas características del turismo en Argentina'' (MINTUR, 2007)}, donde se propone un estudio ``desde el punto de vista de la oferta, es decir estudiando el empleo en las ramas características del turismo'' utilizando la Encuesta Permanente de Hogares (EPH) como fuente de información. Para la definición de las RCT toma como referencia la lista propuesta por la OMT, a fin de mantener la comparabilidad internacional.
Partiendo de dicha clasificación, se estima el volumen de empleo en las RCT en Argentina, dedicando un apartado a describir las características sociodemográficas del empleo en el sector. Sin embargo, cabe señalar que la EPH sólo cubre algo más del 60\% de la población nacional, puesto que se releva en los 31 grandes aglomerados del país. Por otro lado, la desagregación de la información por rama de actividad no permite discriminar, en muchos casos, componentes turísticos y no turísticos dentro de un mismo código de actividad (por ejemplo, ``Transporte terrestre de pasajeros'' incluye tanto el urbano como el interurbano, cuando es este último el único que debería considerarse como turístico).
Para analizar la estacionalidad del sector, utiliza la Encuesta de Ocupación Hotelera (EOH) para estimar las variaciones estacionales de los puestos de trabajo en una rama específica, los hoteles. Cabe indicar que en este caso los resultados obtenidos se refieren a puestos de trabajo, en lugar de personas ocupadas (como corresponde a las estimaciones sobre la EOH).
Por otro lado, la Cámara Argentina de Turismo utiliza un procedimiento de cálculo desde una perspectiva de la Demanda para estimar el empleo, como lo explica en su \emph{``Informe económico anual sobre la actividad de viajes y turismo (2008)''}. Utiliza el Método de Coeficientes Fijos (ratios) para estimar el nivel de empleo por rama y categoría, donde se aproxima a la cantidad de empleo generada por la actividad económica de turismo y viajes bajo el supuesto de construcción de coeficientes fijos.
En este sentido, los coeficientes recorren transversalmente las actividades económicas de cuentas nacionales vinculadas de forma directa e indirecta con el sector turismo para estimar su participación en el Producto Interno Bruto (PIB). En particular, a partir de estimaciones de la demanda que realizan los turistas de productos ofrecidos por los distintos sectores económicos se determinan los coeficientes de participación del turismo sobre el valor agregado bruto de cada una de estas actividades. Esto implica que, por ejemplo, al sector hotelero, en el cual el sector turismo tiene una importancia significativa, se le aplique un coeficiente de 100\% y, por el contrario, a la industria manufacturera, que produce tanto para visitantes como no visitantes, se le asigne un coeficiente sustancialmente menor.
Cada uno de estos coeficientes mide el porcentaje del valor agregado de cada actividad económica que pertenece al turismo. No obstante lo interesante de este ejercicio, es cuestionable su validez empírica por la dificultad de obtener sus coeficientes con cierta confiabilidad a partir de la información hoy disponible en el país.

\textbf{AGREGAR CST??}

  \bibliography{book.bib,packages.bib}

\end{document}
